%%AMC:latex_engine=xelatex
\RequirePackage[l2tabu, orthodox]{nag}
\RequirePackage{silence}\WarningFilter{fmtcount}{\ordinal already defined use \FCordinal instead}
\documentclass[version=last, pagesize, twoside=semi, DIV=calc, 12pt, a4paper, english, french]{scrartcl}
\usepackage{environ}
\usepackage[completemulti, francais]{automultiplechoice}
%INSTALL

%avoids a warning
\usepackage[log-declarations=false]{xparse}
\usepackage{fontspec} %font selecting commands
\usepackage{xunicode}
%warn about missing characters
\tracinglostchars=2

%REDAC
\usepackage{booktabs}
\usepackage{calc}

\usepackage{mathtools} %load this before babel!
	\mathtoolsset{showonlyrefs,showmanualtags}

\usepackage{babel}
%suppresses the warning about frenchb not modifying the captions (“—” to “:” in “Figure 1 – Legend”).
	\frenchbsetup{AutoSpacePunctuation=false,SuppressWarning=false}

\usepackage[super]{nth}
\usepackage{listings} %typeset source code listings
	\lstset{language=XML,tabsize=2,literate={"}{{\tt"}}1,captionpos=b}
\usepackage[nolist,smaller,printonlyused]{acronym}%,smaller option produces warnings from relsize in some cases, it seems.
\usepackage[nodayofweek]{datetime}%must be loaded after the babel package
\usepackage{xspace}
\usepackage{hyperref}% option pdfusetitle must be introduced here, not in hypersetup.
%breaklinks makes links on multiple lines into different PDF links to the same target.
%colorlinks (false): Colors the text of links and anchors. The colors chosen depend on the the type of link. In spite of colored boxes, the colored text remains when printing.
%linkcolor=black: this leaves other links in colors, e.g. refs in green, don't print well.
%pdfborder (0 0 1, set to 0 0 0 if colorlinks): width of PDF link border
%hidelinks
\hypersetup{breaklinks,bookmarksopen,colorlinks=true,urlcolor=blue,linkcolor=,hyperfigures=true}
% hyperref doc says: Package bookmark replaces hyperref’s bookmark organization by a new algorithm (...) Therefore I recommend using this package.
\usepackage{bookmark}

% center floats by default, but do not use with float
% \usepackage{floatrow}
% \makeatletter
% \g@addto@macro\@floatboxreset\centering
% \makeatother
\usepackage{ragged2e} %new com­mands \Cen­ter­ing, \RaggedLeft, and \RaggedRight and new en­vi­ron­ments Cen­ter, FlushLeft, and FlushRight, which set ragged text and are eas­ily con­fig­urable to al­low hy­phen­ation (the cor­re­spond­ing com­mands in LaTeX, all of whose names are lower-case, pre­vent hy­phen­ation al­to­gether). 
\usepackage{siunitx} %[expproduct=tighttimes, decimalsymbol=comma]
\sisetup{detect-all}% to detect e.g. when in math mode (use a math font)
\usepackage{braket} %for \Set
\usepackage{natbib}

\usepackage{amsmath,amsthm}
% \usepackage{amsfonts} %not required?
% \usepackage{dsfont} %for what?
%unicode-math overwrites the following commands from the mathtools package: \dblcolon, \coloneqq, \Coloneqq, \eqqcolon. Using the other colon-like commands from mathtools will lead to inconsistencies. Plus, Using \overbracket and \underbracke from mathtools package. Use \Uoverbracket and \Uunderbracke for original unicode-math definition.
%use exclusively \mathbf and choose math bold style below.
\usepackage[warnings-off={mathtools-colon, mathtools-overbracket}, bold-style=ISO]{unicode-math}

\defaultfontfeatures{
	Fractions=On,
	Mapping=tex-text% to turn "--" into dashes, useful for bibtex%%
}
\defaultfontfeatures[\rmfamily, \sffamily]{
	Fractions=On,
	Mapping=% to leave " alone (disable the default mapping tex-text; requires loading the font afterwards?)
}
\newfontfamily\xitsfamily{XITS}
\newfontfamily\texgyretermesfamily{TeX Gyre Termes}
\newfontfamily\texgyreherosfamily{TeX Gyre Heros}
\newfontfamily\lmfamily{Latin Modern Roman}
\setmainfont{Latin Modern Roman}
\setsansfont{Latin Modern Sans}
\setmonofont{Latin Modern Mono}
\defaultfontfeatures{}%disable default font features to avoid warnings with math fonts.
\setmathfont{XITS Math}
\setmathfont[range={\mathcal,\mathbfcal},StylisticSet=1]{XITS Math}

\usepackage{cleveref}% cleveref should go "laster" than hyperref
%GRAPHICS
\usepackage{pgf}
\usepackage{pgfplots}
	\usetikzlibrary{matrix,fit,plotmarks,calc,trees,shapes.geometric,positioning,plothandlers}
\pgfplotsset{compat=1.11}
\usepackage{graphicx}

\graphicspath{{graphics/},{graphics-dm/}}
\DeclareGraphicsExtensions{.pdf}

%HACKING
\usepackage{printlen}
\uselengthunit{mm}
% 	\newlength{\templ}% or LenTemp?
% 	\setlength{\templ}{6 pt}
% 	\printlength{\templ}
\usepackage{etoolbox} %for addtocmd command
\usepackage{scrhack}% load at end. Corrects a bug in float package, which is outdated but might be used by other packages
\usepackage{xltxtra} %somebody said that this is loaded by fontspec, but does not seem correct: if not loaded explicitly, does not appear in the log and \showhyphens is not corrected.

%Beamer-specific
%ADD
\usepackage{appendixnumberbeamer}
%\setbeamersize{text margin left=0.1cm, text margin right=0.1cm} 
\setbeamertemplate{navigation symbols}{}
\usetheme{BrusselsBelgium}
\usefonttheme{professionalfonts}


\newcommand{\R}{ℝ}
\newcommand{\N}{ℕ}
\newcommand{\Z}{ℤ}
\newcommand{\card}[1]{\lvert{#1}\rvert}
\newcommand{\powerset}[1]{\mathscr{P}(#1)}%\mathscr rather than \mathcal: scr is rounder than cal (at least in XITS Math).
\newcommand{\suchthat}{\;\ifnum\currentgrouptype=16 \middle\fi|\;}
%\newcommand{\Rplus}{\reels^+\xspace}

\AtBeginDocument{%
	\renewcommand{\epsilon}{\varepsilon}
% we want straight form of \phi for mathematics, as recommended in UTR #25: Unicode support for mathematics.
%	\renewcommand{\phi}{\varphi}
}

% with amssymb, but I don’t want to use amssymb just for that.
% \newcommand{\restr}[2]{{#1}_{\restriction #2}}
%\newcommand{\restr}[2]{{#1\upharpoonright}_{#2}}
\newcommand{\restr}[2]{{#1|}_{#2}}%sometimes typed out incorrectly within \set.
%\newcommand{\restr}[2]{{#1}_{\vert #2}}%\vert errors when used within \Set and is typed out incorrectly within \set.
\DeclareMathOperator*{\argmax}{arg\,max}
\DeclareMathOperator*{\argmin}{arg\,min}


%ARG TH
\newcommand{\AF}{\mathcal{AF}}
\newcommand{\labelling}{\mathcal{L}}
\newcommand{\labin}{\textbf{in}\xspace}
\newcommand{\labout}{\textbf{out}}
\newcommand{\labund}{\textbf{undec}\xspace}
\newcommand{\nonemptyor}[2]{\ifthenelse{\equal{#1}{}}{#2}{#1}}
\newcommand{\gextlab}[2][]{
	\labelling{\mathcal{GE}}_{(#2, \nonemptyor{#1}{\ibeatsr{#2}})}
}
\newcommand{\allargs}{A^*}
\newcommand{\args}{A}
\newcommand{\ar}{a}

%MCDA+Arg
\newcommand{\dm}{d}
\newcommand{\ileadsto}{\leadsto}
\newcommand{\mleadsto}[1][\eta]{\leadsto_{#1}}
\newcommand{\ibeats}{\vartriangleright}
\newcommand{\mbeats}[1][\eta]{\vartriangleright_{#1}}


%MISC
\newcommand{\lequiv}{\Vvdash}
\newcommand{\weightst}{W^{\,t}}

%MCDA classical
\newcommand{\crits}{\mathcal{J}}
\newcommand{\altspace}{\mathbb{A}}
\newcommand{\alts}{A}

%Sorting
\newcommand{\cats}{\mathcal{C}}
\newcommand{\catssubsets}{2^\cats}
\newcommand{\catgg}{\vartriangleright}
\newcommand{\catll}{\vartriangleleft}
\newcommand{\catleq}{\trianglelefteq}
\newcommand{\catgeq}{\trianglerighteq}
\newcommand{\alttoc}[2][x]{(#1 \xrightarrow{} #2)}
\newcommand{\alttocat}[3]{(#2 \xrightarrow{#1} #3)}
\newcommand{\alttoI}{(x \xrightarrow{} \left[\underline{C_x}, \overline{C_x}\right])}
\newcommand{\alttocatdm}[3][t]{\left(#2 \thinspace \raisebox{-3pt}{$\xrightarrow{#1}$}\thinspace #3\right)}
\newcommand{\alttocatatleast}[2]{\left(#1 \thinspace \raisebox{-3pt}{$\xrightarrow[]{≥}$}\thinspace #2\right)}
\newcommand{\alttocatatmost}[2]{\left(#1 \thinspace \raisebox{-3pt}{$\xrightarrow[]{≤}$}\thinspace #2\right)}

\newcommand{\source}{\scriptsize}
\newcommand{\commentOC}[1]{{\selectlanguage{french}{\todo{OC : #1}}}}
%Or: \todo[color=green!40]

%this probably requires outdated float package, see doc KomaScript for an alternative.
% \newfloat{program}{t}{lop}
% \floatname{program}{PM}

%\crefname{axiom}{axiom}{axioms}%might be needed for workaround bug in cref when defining new theorems?

%\ifdefined\theorem\else
%\newtheorem{theorem}{\iflanguage{english}{Theorem}{Théorème}}
%\fi

%which line breaks are chosen: accept worse lines, therefore reducing risk of overfull lines. Default = 200
\tolerance=2000
%accept overfull hbox up to...
\hfuzz=2cm
%reduces verbosity about the bad line breaks
\hbadness 5000
%sloppy sets tolerance to 9999
\apptocmd{\sloppy}{\hbadness 10000\relax}{}{}

% WRITING
%\newcommand{\ie}{i.e.\@\xspace}%to try
%\newcommand{\eg}{e.g.\@\xspace}
%\newcommand{\etal}{et al.\@\xspace}
\newcommand{\ie}{i.e.\ }
\newcommand{\eg}{e.g.\ }
\newcommand{\mkkOK}{\checkmark}%\color{green}{\checkmark}
\newcommand{\mkkREQ}{\ding{53}}%requires pifont?%\color{green}{\checkmark}
\newcommand{\mkkNO}{}%\text{\color{red}{\textsf{X}}}

\makeatletter
\newcommand{\boldor}[2]{%
	\ifnum\strcmp{\f@series}{bx}=\z@
		#1%
	\else
		#2%
	\fi
}
\newcommand{\textstyleElProm}[1]{\boldor{\MakeUppercase{#1}}{\textsc{#1}}}
\makeatother
\newcommand{\electre}{\textstyleElProm{Électre}\xspace}
\newcommand{\electreIv}{\textstyleElProm{Électre Iv}\xspace}
\newcommand{\electreIV}{\textstyleElProm{Électre IV}\xspace}
\newcommand{\electreIII}{\textstyleElProm{Électre III}\xspace}
\newcommand{\electreTRI}{\textstyleElProm{Électre Tri}\xspace}
% \newcommand{\utadis}{\texorpdfstring{\textstyleElProm{utadis}\xspace}{UTADIS}}
% \newcommand{\utadisI}{\texorpdfstring{\textstyleElProm{utadis i}\xspace}{UTADIS I}}

%TODO
% \newcommand{\textstyleElProm}[1]{{\rmfamily\textsc{#1}}} 


\newlength{\GraphsNodeSep}
\setlength{\GraphsNodeSep}{7mm}

% MCDA Drawing Sorting
\newlength{\MCDSCatHeight}
\setlength{\MCDSCatHeight}{6mm}
\newlength{\MCDSAltHeight}
\setlength{\MCDSAltHeight}{4mm}
%separation between two vertical alts
\newlength{\MCDSAltSep}
\setlength{\MCDSAltSep}{2mm}
\newlength{\MCDSCatWidth}
\setlength{\MCDSCatWidth}{3cm}
\newlength{\MCDSEvalRowHeight}
\setlength{\MCDSEvalRowHeight}{6mm}
\newlength{\MCDSAltsToCatsSep}
\setlength{\MCDSAltsToCatsSep}{1.5cm}
\newcounter{MCDSNbAlts}
\newcounter{MCDSNbCats}
\newlength{\MCDSArrowDownOffset}
\setlength{\MCDSArrowDownOffset}{0mm}

\tikzset{/Graphs/dot/.style={
	shape=circle, fill=black, inner sep=0, minimum size=1mm
}}
\tikzset{/MC/D/S/alt/.style={
	shape=rectangle, draw=black, inner sep=0, minimum height=\MCDSAltHeight, minimum width=2.5cm, anchor=north east
}}
\tikzset{MC/D/S/pref/.style={
	shape=ellipse, draw=gray, thick
}}
\tikzset{/MC/D/S/cat/.style={
	shape=rectangle, draw=black, inner sep=0, minimum height=\MCDSCatHeight, minimum width=\MCDSCatWidth, anchor=north west
}}
\tikzset{/MC/D/S/evals matrix/.style={
	matrix, row sep=-\pgflinewidth, column sep=-\pgflinewidth, nodes={shape=rectangle, draw=black, inner sep=0mm, text depth=0.5ex, text height=1em, minimum height=\MCDSEvalRowHeight, minimum width=12mm}, nodes in empty cells, matrix of nodes, inner sep=0mm, outer sep=0mm, row 1/.style={nodes={draw=none, minimum height=0em, text height=, inner ysep=1mm}}
}}

% Beliefs
\tikzset{/Beliefs/D/S/attacker/.style={
	shape=rectangle, draw, minimum size=8mm
}}
\tikzset{/Beliefs/D/S/supporter/.style={
	shape=circle, draw
}}

\newcommand{\tikzmark}[1]{%
	\tikz[overlay, remember picture, baseline=(#1.base)] \node (#1) {};%
}


\begin{acronym}
\acro{AMCD}{Aide Multicritère à la Décision}
\acro{ASA}{Argument Strength Assessment}
\acro{DA}{Decision Analysis}
\acro{DM}{Decision Maker}
\acro{DPr}{Deliberated Preferences}
\acro{DRSA}{Dominance-based Rough Set Approach}
\acro{DSS}{Decision Support Systems}
\acrodefplural{DSS}{Decision Support Systems}
% \newacroplural{DSS}[DSSes]{Decision Support Systems}
\acro{EJOR}{European Journal of Operational Research}
\acro{LNCS}{Lecture Notes in Computer Science}
\acro{MCDA}{Multicriteria Decision Aid}
\acro{MIP}{Mixed Integer Program}
\acro{NCSM}{Non Compensatory Sorting Model}
\acro{PL}{Programme Linéaire}
\acro{PLNE}{Programme Linéaire en Nombres Entiers}
\acro{PM}{Programme Mathématique}
\acro{MP}{Mathematical Program}
\acro{MIP}{Mixed Integer Program}
% \newacroplural{PM}{Programmes Mathématiques}
%acrodefplural since version 1.35, my debian has \ProvidesPackage{acronym}[2009/01/25, v1.34, Support for acronyms (Tobias Oetiker)]
\acrodefplural{PM}{Programmes Mathématiques}
\acro{PMML}{Predictive Model Markup Language}
\acro{RESS}{Reliability Engineering \& System Safety}
\acro{SMAA}{Stochastic Multicriteria Acceptability Analysis}
\acro{URPDM}{Uncertainty and Robustness in Planning and Decision Making}
\acro{XML}{Extensible Markup Language}
\end{acronym}


\begin{document}
\title{QCM Test}
\author{}
\date{\vspace{-2em}\formatdate{31}{05}{2017}}
\AMCopenOpts{dots=false, lines=5, scan=false}

\AMCrandomseed{1237893}

\element{wholegroup}{
\begin{questionmult}{git}
	Cocher toutes les affirmations correctes concernant HEAD dans un dépôt git quelconque.
	\begin{reponses}
		\mauvaise{HEAD pointe, généralement mais pas toujours, vers un commit (directement ou indirectement)}
		\bonne{HEAD pointe toujours vers un commit (directement ou indirectement)}
		\bonne{HEAD pointe, généralement mais pas toujours, vers une branche (directement ou indirectement)}
		\mauvaise{HEAD pointe toujours vers une branche (directement ou indirectement)}
		\mauvaise{HEAD représente un sous-ensemble de l’arbre des commits}
		\mauvaise{HEAD représente l’équivalent distant de l’historique sauvegardé localement par git, lorsqu’un serveur distant est configuré}
	\end{reponses}
\end{questionmult}
}

\element{wholegroup}{
\begin{question}{equals}
	En Java, \texttt{"a".equals("b");}
	\begin{reponses}
		\mauvaise{ne compile pas}
		\mauvaise{compile, mais produit une erreur à l’exécution}
		\bonne{exécute la méthode equals définie dans la classe String}
		\mauvaise{exécute la méthode equals définie dans la classe Object}
	\end{reponses}
\end{question}
}

\element{wholegroup}{
\begin{questionmult}{contrat}
	Considérer une méthode d’en-tête suivant, qui retourne $\sqrt{a + b}$.
	
	\texttt{public double sqrtSum(double a, double b);}

	Le principe de la programmation par contrat requiert (cocher tout ce qui s’applique) de
	\begin{reponses}
		\bonne{documenter l’en-tête des méthodes pour indiquer les contraintes sur les paramètres}
		\bonne{documenter l’en-tête des méthodes pour indiquer les garanties sur les résultats}
		\mauvaise{documenter l’en-tête des méthodes pour indiquer le fonctionnement détaillé de la méthode de façon suffisamment claire pour que l’utilisateur de la méthode sache comment la méthode procède pour calculer son résultat, sans devoir lire le code source (par exemple en indiquant l’algorithme utilisé pour le calcul, ou les méthodes appelées)}
		\mauvaise{affecter tout résultat intermédiaire à une variable, dans le corps de la méthode}
		\mauvaise{logger les valeurs des paramètres reçus et de la valeur renvoyée}
		\mauvaise{utiliser \texttt{assert} pour avertir l’utilisateur final lorsqu’il entre des valeurs incorrectes dans l’interface graphique (si une interface graphique est utilisée)}
	\end{reponses}
\end{questionmult}
}

\newsavebox\lstboxexc
\begin{lrbox}{\lstboxexc}
\begin{lstlisting}
System.out.println("Coucou1");
Exception e = new IllegalArgumentException();
System.out.println("Coucou2");
throw e;
System.out.println("Coucou3");
\end{lstlisting}
\end{lrbox}  

\element{wholegroup}{
\begin{question}{exc}
	Considérer le code Java suivant (supposé constituer une méthode).

	\usebox\lstboxexc
	
	Cette méthode
	\begin{reponses}
		\mauvaise{ne compile pas}
		\mauvaise{compile et affiche \texttt{Coucou1} puis lance une exception, et n’affiche rien d’autre}
		\bonne{compile et affiche \texttt{Coucou1} puis \texttt{Coucou2} puis lance une exception, et n’affiche rien d’autre}
		\mauvaise{compile et affiche \texttt{Coucou1} puis \texttt{Coucou2} puis lance une exception puis affiche \texttt{Coucou3}}
		\mauvaise{compile et affiche \texttt{Coucou1} puis \texttt{Coucou2} puis lance une exception puis affiche \texttt{Coucou3} si et seulement si l’appelant de la méthode "attrape" (\texttt{catch}) l’exception lancée}
		\mauvaise{compile et affiche \texttt{Coucou1} puis \texttt{Coucou2} puis lance une exception puis affiche \texttt{Coucou3} si et seulement si l’appelant de la méthode n’attrape pas l’exception lancée}
	\end{reponses}
\end{question}
}

\newsavebox\lstboxinterfacescorps
\begin{lrbox}{\lstboxinterfacescorps}%
\begin{lstlisting}[xleftmargin=3.4pt, xrightmargin=13.4pt]
public interface MyInterface {
	public void myFirstMethod() {
		System.out.println("Coucou.");
	}
}
public class MyClass implements MyInterface {
	public void myMethod() {
		System.out.println("Coucou.");
	}
}
\end{lstlisting}%
\end{lrbox}  

\element{wholegroup}{
\begin{question}{interfacescorps}
	Le code suivant compile-t-il ? (On suppose que le code est placé dans des fichiers nommés de façon adéquate et que les imports éventuels sont corrects.)
	
	\usebox\lstboxinterfacescorps

	\begin{reponses}
		\mauvaise{Oui}
		\bonne{Non}
	\end{reponses}
\end{question}
}

\element{wholegroup}{
\begin{question}{maven}
	Soit un projet configuré de façon classique utilisant Maven, structuré en sous-répertoires et en fichiers comme suit :
	\begin{itemize}
		\item Racine, contenant pom.xml
		\item Racine/src/main/java/rep1, contenant AFile.java et AFile.txt
		\item Racine/src/test/java, contenant AnotherFile.java
	\end{itemize}
En supposant que le projet compile, quelles classes et ressources se retrouveront dans le classpath lors de l’exécution des tests ? (Une seule réponse.)
	\begin{reponses}
		\mauvaise{pom.xml, les classes AFile et AnotherFile, et le fichier ressource AFile.txt.}
		\bonne{Les classes AFile et AnotherFile, et le fichier ressource AFile.txt.}
		\mauvaise{Les classes AFile et AnotherFile.}
		\mauvaise{La classe AnotherFile.}
		\mauvaise{Le fichier ressource AFile.txt.}
		\mauvaise{Rien.}
	\end{reponses}
\end{question}
}

\element{wholegroup}{
\begin{questionmult}{swt}
	SWT est (cocher toutes les affirmations correctes)
	\begin{reponses}
		\bonne{une bibliothèque non incluse en standard dans Java}
		\mauvaise{une bibliothèque incluse en standard dans Java}
		\mauvaise{un autre nom pour Swing}
		\mauvaise{une bibliothèque allant souvent de pair avec Swing}
		\bonne{une bibliothèque allant souvent de pair avec JFace}
	\end{reponses}
\end{questionmult}
}

\newsavebox\lstboxcreat
\begin{lrbox}{\lstboxcreat}
\begin{lstlisting}[escapechar=\$]
public class MyClass {
	public static void printOut(Set<String> s) {
		// nothing
	}
	public static void main(String args[]) { 
		$…$
	}
}
\end{lstlisting}
\end{lrbox}  

\element{wholegroup}{
\begin{questionmult}{creat}
	Considérons la classe suivante.

	\usebox\lstboxcreat
	
	Quels codes sont corrects, en supposant qu’ils soient insérés dans la méthode \texttt{main} ci-dessus à la place des points de suspension ?
	\begin{reponses}
		\bonne{\texttt{Set<String> strs = new HashSet<>();} \texttt{printOut(strs);}}
		\bonne{\texttt{HashSet<String> strs = new HashSet<>(); printOut(strs);}}
		\mauvaise{\texttt{HashSet<Integer> strs = new HashSet<>(); printOut(strs);}}
		\bonne{\texttt{HashSet<String> strs = new HashSet<String>(); printOut(strs);}}
		\mauvaise{\texttt{HashSet<String> strs = new HashSet<Integer>(); printOut(strs);}}
	\end{reponses}
\end{questionmult}
}

\newsavebox\lstboxgener
\begin{lrbox}{\lstboxgener}
\begin{lstlisting}
public interface MyInterface<V> {
	public V myFirstMethod();
}

public class MyClass<V> implements MyInterface<Integer> {
	public void myMethod() {
		System.out.println("Coucou.");
	}
	public Integer myFirstMethod() {
		return null;
	}
}
\end{lstlisting}
\end{lrbox}  

\element{wholegroup}{
\begin{question}{gener}
	Le code suivant compile-t-il ? (On suppose que le code est placé dans des fichiers nommés de façon adéquate et que les imports éventuels sont corrects.)
	
	\usebox\lstboxgener
	
	\begin{reponses}
		\bonne{Oui}
		\mauvaise{Non}
	\end{reponses}
\end{question}
}

\newsavebox\lstboxhashcode
\begin{lrbox}{\lstboxhashcode}
\begin{lstlisting}
public class MyClass {
	private String lastName;
	private String firstName;
	public boolean equals(Object o) {
		if (!o instanceof MyClass) return false;
		return Objects.equals(firstName, ((MyClass)o).firstName);
	}
	public int hashcode() {
		return Objects.hash(firstName, lastName);
	}
}
\end{lstlisting}
\end{lrbox}  

\element{wholegroup}{
\begin{question}{hashcode}
	L’implémentation suivante respecte-t-elle bien ce qui est exigé d’une fonction de hachage, relativement à l’implémentation de \texttt{equals} fournie ? Voici la javadoc des méthodes invoquées : 
	\begin{itemize}
		\item \texttt{static boolean equals(Object a, Object b)} : “Returns true if the arguments are equal to each other and false otherwise”
		\item \texttt{static int hash(Object... values)} : “Generates a hash code for a sequence of input values”
	\end{itemize}

	\usebox\lstboxhashcode
	
	\begin{reponses}
		\mauvaise{Oui}
		\bonne{Non}
	\end{reponses}
\end{question}
}

\begin{copieexamen}[2]
\maketitle
\thispagestyle{fancy}

Une réponse est considérée comme correcte \emph{seulement} quand elle est entièrement correcte. Sinon, elle est incorrecte. Par exemple, si une question demande de cocher toutes les affirmations vraies, qu’il y a trois affirmations vraies sur quatre, et que vous avez coché deux des trois affirmations vraies, la réponse est considérée comme incorrecte.

Une réponse correcte rapporte un point. Une réponse incorrecte coute un demi-point, sauf s’il n’y a que deux réponses possibles (questions de type vrai ou faux), auquel cas une réponse incorrecte coute un point.

Par exemple, si vous avez 5 réponses correctes, 2 réponses incorrectes, et avez laissé 3 questions sans réponse, dans un questionnaire sans questions vrai ou faux, vous obtenez 4/10.

\vspace{1em}
\champnom{%
	\fbox{%
		\parbox{\dimexpr\textwidth-2\fboxsep-2\fboxrule}{%
			\vspace{1em}
			\begin{description}
				\item[Nom] \dotfill
				\item[Prénom] \vspace{1em}\dotfill
			\end{description}
		}%
	}
}
\vspace{1em}

% Desired: v or e (no answer or incoherent) ⇒ 0; all correct ⇒ 1; some incorrect ⇒ − 0.5.
%fails, only counts whether the number of checked answers are correct, not the right ones
%\baremeDefautM{formula=(NB==NBC? 1 : -0.5), v=0, e=0}
%the following works if you substract 0.5 to each question.
%\baremeDefautM{v=0.5, e=0.5, mz=1.5}
%doesn’t work, seems like mz unaffected by d
%\baremeDefautM{v=0, e=0, mz=3.5, d=4}
%N = nb of answers checked
\baremeDefautM{v=0, e=0, m=-100, b=1, p=-0.5, d=1-N}
\baremeDefautS{v=0, e=0, m=-1, b=1, p=-1}

%rather use setdefaultgroupmode, but my version might be too old.
\shufflegroup{wholegroup}
\restituegroupe{wholegroup}

%\vspace{1cm}\mbox{}%rempli la page (évite tassement en bas quand underful vbox)
%\def\AMCbeginQuestion#1#2{\par #2\hspace*{1em}}%évite d’afficher Question #1
\end{copieexamen}
\end{document}
