\newcommand{\source}{\scriptsize}
\newcommand{\commentOC}[1]{{\selectlanguage{french}{\todo{OC : #1}}}}
%Or: \todo[color=green!40]

%this probably requires outdated float package, see doc KomaScript for an alternative.
% \newfloat{program}{t}{lop}
% \floatname{program}{PM}

%\crefname{axiom}{axiom}{axioms}%might be needed for workaround bug in cref when defining new theorems?

%\ifdefined\theorem\else
%\newtheorem{theorem}{\iflanguage{english}{Theorem}{Théorème}}
%\fi

%which line breaks are chosen: accept worse lines, therefore reducing risk of overfull lines. Default = 200
\tolerance=2000
%accept overfull hbox up to...
\hfuzz=2cm
%reduces verbosity about the bad line breaks
\hbadness 5000
%sloppy sets tolerance to 9999
\apptocmd{\sloppy}{\hbadness 10000\relax}{}{}

% WRITING
%\newcommand{\ie}{i.e.\@\xspace}%to try
%\newcommand{\eg}{e.g.\@\xspace}
%\newcommand{\etal}{et al.\@\xspace}
\newcommand{\ie}{i.e.\ }
\newcommand{\eg}{e.g.\ }
\newcommand{\mkkOK}{\checkmark}%\color{green}{\checkmark}
\newcommand{\mkkREQ}{\ding{53}}%requires pifont?%\color{green}{\checkmark}
\newcommand{\mkkNO}{}%\text{\color{red}{\textsf{X}}}

\makeatletter
\newcommand{\boldor}[2]{%
	\ifnum\strcmp{\f@series}{bx}=\z@
		#1%
	\else
		#2%
	\fi
}
\newcommand{\textstyleElProm}[1]{\boldor{\MakeUppercase{#1}}{\textsc{#1}}}
\makeatother
\newcommand{\electre}{\textstyleElProm{Électre}\xspace}
\newcommand{\electreIv}{\textstyleElProm{Électre Iv}\xspace}
\newcommand{\electreIV}{\textstyleElProm{Électre IV}\xspace}
\newcommand{\electreIII}{\textstyleElProm{Électre III}\xspace}
\newcommand{\electreTRI}{\textstyleElProm{Électre Tri}\xspace}
% \newcommand{\utadis}{\texorpdfstring{\textstyleElProm{utadis}\xspace}{UTADIS}}
% \newcommand{\utadisI}{\texorpdfstring{\textstyleElProm{utadis i}\xspace}{UTADIS I}}

%TODO
% \newcommand{\textstyleElProm}[1]{{\rmfamily\textsc{#1}}} 

\newcommand{\menuit}{\emph}

\newbool{refAPIIsAnnotation}
\pgfkeys{
	/refAPI/.is family,
	/refAPI/.cd,
	base url/.estore in = \refAPIBaseUrl,
	@/.is if = refAPIIsAnnotation,
	prefix/.store in = \refAPIPrefix,
	suffix/.estore in = \refAPISuffix,
	default/.style = {
		base url = ,
		@ = false,
		prefix = \texttt,
		suffix = ,
	},
}

%Usage: \refAPIParseRef{javax.persistence/EntityManager} ; \refAPIParseRef{javax.persistence/PersistenceContextType\#EXTENDED}
%Defines: \refAPIFQName (before #) ; \refAPIField (after #) ; \refAPIPackageSlashes (FQName before / with slashes instead of dots) ; \refAPIClass (FQName after /)
\newcommand{\refAPIParseRef}[1]{%
	\IfSubStr{#1}{\#}{%
		\StrBefore{#1}{\#}[\refAPIFQName]%
		\StrBehind{#1}{\#}[\refAPIField]%
	}{%
		\edef\refAPIFQName{#1}%
		\edef\refAPIField{}%
	}%
	\StrBefore{\refAPIFQName}{/}[\refAPIPackage]%
	\StrBehind{\refAPIFQName}{/}[\refAPIClass]%
	\StrSubstitute{\refAPIPackage}{.}{/}[\refAPIPackageSlashes]%
}

%Usage: \jeeref{javax.persistence/EntityManager} ; \jeeref[@]{javax.persistence/PersistenceContextType\#EXTENDED}
\newcommand{\jeeref}[2][]{\refAPIParseRef{#2}\jrefAPI{base url = https://docs.oracle.com/javaee/7/api/, #1}}
\newcommand{\jseref}[2][]{\refAPIParseRef{#2}\jrefAPI{base url = https://docs.oracle.com/javase/8/docs/api/, #1}}

%Accepts keys: @, prefix, suffix. Requires key: base url. Expects defined: \refAPIField ; \refAPIPackageSlashes ; \refAPIClass
\newcommand{\jrefAPI}[1]{%
	\pgfkeys{/refAPI/.cd, default, #1}%
	\ifbool{refAPIIsAnnotation}{\edef\refAPIAnnot{@}}{\edef\refAPIAnnot{}}%
	\IfEq{\refAPIField}{}{%
		{\refAPIPrefix{%
			\href{%
				\refAPIBaseUrl\refAPIPackageSlashes/\refAPIClass.html\#\refAPIField%
			}{%
				\refAPIAnnot\refAPIClass%
			}%
		}\refAPISuffix}%
	}{%
		{\refAPIPrefix{%
			\refAPIAnnot\refAPIClass.%
			\href{%
				\refAPIBaseUrl\refAPIPackageSlashes/\refAPIClass.html%
			}{%
				\refAPIField%
			}%
		}\refAPISuffix}%
	}%
}

